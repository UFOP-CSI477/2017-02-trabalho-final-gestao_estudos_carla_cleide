\documentclass[10pt,a4paper,article]{abntex2}
\usepackage[utf8]{inputenc}
\usepackage{amsmath}
\usepackage{amsfonts}
\usepackage{amssymb}
\usepackage{url}

% Nome dos autores do trabalho
\title{CSI477-2017-02 -- Proposta de Trabalho Final}
\author{Grupo: Carla Sanches Nere dos Santos \& Cleide Helena Almeida Silva}
\begin{document}

	\maketitle

	% Descrever um resumo sobre o trabalho.
	\begin{abstract}
		O objetivo deste documento é apresentar uma proposta para o trabalho a ser desenvolvido na disciplina CSI477 -- Sistemas WEB I. \newline
		
		O trabalho consiste num website em que os usuários poderão organizar seus cursos, separando-os em disciplinas e seus eventos relacionados.\newline
	
		
		Datas de provas não serão mais perdidas com a interface do calendário, que além de marcar eventos, envia notificações quando eles estão perto de começar. Gráficos de acompanhamento de estudo mostrarão a evolução do desempenho das disciplinas e o estudante poderá ter mais autonomia ao gerenciar suas faltas.
	\end{abstract}

	% Apresentar o tema.
	\section{Tema}

		O trabalho final tem como tema o desenvolvimento de um website para o gerenciamento de estudos. Ele poderá abordar tanto as disciplinas de escola e faculdade quanto cursos extras realizados pelo usuário.

	% Descrever e limitar o escopo da aplicação.
	\section{Escopo}

		Este projeto terá as seguintes funcionalidades:
		
		\begin{itemize}
		    \item Calendário para marcar datas de provas, tarefas e atividades avaliativas;
		    \item Espaço de organização das disciplinas, com links para as páginas de inserção de notas, arquivos e eventos relacionados à cada disciplina.
		    \item Formulário de cadastro de dados do usuário.
		    \item Gráfico de acompanhamento de disciplina por nota e por dedicação.
		    \item Previsão de notas de acordo com a média estabelecida
		    \item Alerta de eventos por prioridade de estudo estabelecida pelo usuário.
		    \item Exibição da agenda de tarefas do dia com opção de marcar o que foi feito.
		    \item Possibilidade de adição de tarefas sem marcar a data no calendário.
		    \item Possibilidade de desativar notificações por disciplina.
		    \item Contagem de dias de estudo para eventos de disciplinas.
		    \item Controle de faltas.
		    \item horário de aulas.
		\end{itemize}

	% Apresentar restrições de funcionalidades e de escopo.
	\section{Restrições}

		O objetivo do website será o gerenciamento por parte do usuário, sendo um aplicativo pessoal e focado em organizar os estudos de maneira personalizada. Portanto, as seguintes funcionalidades não serão consideradas:
		
		\begin{itemize}
		    \item Gerenciamento de grupos de estudo.
		    \item Acompanhamento de professores/tutores.
		\end{itemize}

  % Construir alguns protótipos para a aplicação, disponibilizá-los no Github e descrever o que foi considerado.
	\section{Protótipos}
		  Protótipos para as páginas de login, cadastro de usuário e lista de tarefas foram elaborados, e podem ser encontrados em: \url{https://github.com/UFOP-CSI477/2017-02-trabalho-final-gestao_estudos_carla_cleide/tree/master/Prototipos}.
		  
		  Foram consideradas as seguintes interfaces:
		  
		  \begin{itemize}
		      \item Login: através da realização do login, os usuários cadastrados poderão ter acesso aos seus dados personalidados de acordo com suas necessidades, que serão inseridas por ele.
		      
		      
		  
		      
		      \item Lista de tarefas: Nesta lista, o usuário poderá adicionar tarefas relacionadas às suas disciplinas sem ter a necessidade de adicionar uma data limite para a sua realização. Isso evita a sobrecarga das datas do calendário, mantendo as prioridades em seu devido lugar.
		  \end{itemize}

	\section{Repositório}

		O trabalho final terá como repositório principal o seguinte endereço: \url{https://github.com/UFOP-CSI477/2017-02-trabalho-final-gestao_estudos_carla_cleide}.


\end{document}
